\chapter{概述}

\section{项目背景}
本项目依托于龙芯杯提供的FPGA实验平台、Soc工程环境以及基准测试程序,设计并实现了一个部分兼容 MIPS32 体系结构的小端序 CPU,名为\cpuname(\textbf{C}QU \textbf{D}ual \textbf{I}ssue \textbf{M}achine),其能成功通过龙芯杯提供的功能测试、性能测试、系统测试,具有较完善的运算处理、AXI访问、异常处理、中断响应等功能,并能够运行u-boot引导程序、uCore操作系统和Linux操作系统等。

\section{名词解释}
本项目中可能用到的一些名词缩写及其解释如表\ref{table:abbreviation_definition}所示。

\begin{table}[!htbp]
    \centering
    \caption{名词缩写和解释}
    \label{table:abbreviation_definition}
    
    \begin{tabular}{cll}
    \toprule
    \multicolumn{1}{c}{\textbf{名词缩写}} & \multicolumn{1}{c}{\textbf{全称}}                   & \multicolumn{1}{c}{\textbf{解释}} \\ 
    \midrule
    MIPS                               & Microprocessor without Interlocked Pipeline Stages & 无内部互锁流水级的微处理器 \\
    SOC                                & System On a Chip                                   & 片上系统 \\
    CPU                                & Central Processing Unit                            & 中央处理器 \\
    ALU                                & Arithmetic Logic Unit                              & 算数逻辑单元 \\
    GPR                                & General Purpose Register                           & 通用寄存器 \\
    CP0                                & Co-Processor 0                                     & 协处理器0 \\
    BRAM                               & Block Random Access Memory                         & 块随机访问存储器 \\
    FIFO                               & First In First Out                                 & 先进先出 \\
    RAW                                & Read After Write                                   & 写后读 \\
    WAW                                & Write After Write                                  & 写后写 \\
    WAR                                & Write After Read                                   & 读后写 \\
    ASID                               & Address Space Identification                       & 地址空间标识 \\
    \bottomrule
    \end{tabular}
\end{table}

\section{项目概述}
本项目,首先设计了一款双发射五级顺序流水线CPU,\cpuname(\textbf{C}QU \textbf{D}ual \textbf{I}ssue \textbf{M}achine),其支持指令FIFO、分支预测、指令缓存和数据缓存等特殊单元,以提升系统性能,最高CPU主频可达100Mhz,IPC比值达36.227。其中,双发射采用对称双发逻辑以充分保证双发率;五级顺序流水线由取指(Instruction Fetch)、译码(Instruction Decode)、执行(Excute)、访存(Memory access),写回(Write Back)五个阶段组成;指令FIFO可以隔离取指阶段和后续阶段,以实现高效取指的作用;指令缓存和数据缓存均采用二路组相联和突发传输的设计,单路均为4KB以满足伪VIPT对虚拟内存最小页面的要求,其中指令缓存的数据行为64bit以适应双发取指要求,数据缓存的数据行为32bit。\cpuname 还支持U-Boot引导程序,并基于该引导程序,成功运行uCore和Linux操作系统。

此外,为了方便仿真调试,本项目还基于Verilator和gtkWave,开发了支持功能测试、性能测试、系统测试、运行操作系统等系列测试的差分测试工具链,在逐步提交比对时生成对应的波形图。该工具链同时支持指定记录trace的开始时间,以避免生成的trace过大;亦支持双发率、Cache命中率、分支预测成功率的统计,以进行针对性的优化。相较于之前Vivado仿真调试或FPGA上板调试,我们开发的工具链可极大缩短调试时间。