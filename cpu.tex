\chapter{CPU设计方案}

CDIM采用对称双发射顺序执行,共有5级流水。CDIM的数据通路示意图如下,其中红线部分为master path,蓝线部分为slave path。在对称双发射中,master和slave的主要区别在于前者PC小于后者,在处理异常、提交等事宜时会被优先处理。当然,上述的“对称”双发射,只是相对于只支持ALU指令双发的非对称逻辑而言,是对称的,因为slave path支持的指令更多。但严格来讲,并不是绝对对称,在处理跳转指令、例外指令、写TLB指令等会刷新流水线的指令时,需要优先在master处理,这在后续的双发策略中会详细说明。

\section{双发策略}


\section{数据通路设计}
\subsection{取值阶段}
\subsection{译码阶段}
\subsection{执行阶段}
\subsection{访存阶段}
\subsection{写回阶段}

\section{冲突处理}
\subsection{数据冲突}
\subsection{结构冲突}
\subsection{控制冲突}

\section{CP0寄存器设计}

\section{中断和异常}

\section{缓存设计}
